\documentclass[12pt]{article}


\usepackage{geometry}
 \geometry{
 a4paper,
 left=20mm,
 right=20mm,
 top=10mm	
 }

\usepackage{amssymb}
\usepackage[utf8]{inputenc}
\usepackage{graphicx}
\usepackage{amstext}
\usepackage{amsmath}
\usepackage[section]{placeins}


\newtheorem{theorem}{Definice}


\title{BI-ZUM + BI-ZNS - Multiagentní systémy, typy agentů, vlastnosti a požadavky na agenty, organizace multiagentního systému (kooperace, kompetice, koordinace, vyjednávání, komunikace), metoda tabule.}
\author{Vastl Martin}

\begin{document}
\maketitle
\author

Multiagentní systémy je kolekce několika nezávislých agentů umístěných ve sdíleném prostředí, z nichž každý vnímá okolní prostedí, jedná tak, aby dosáhl svých cílů a interaguje s ostatními agenty.

\section{Typy agentů}
Typy agentů dle ZNS:
\begin{itemize}
\item Reaktivní agent - reaguje na podněty z okolí tím způsobem, že má předem danou množinu možných akcí, které může provádět, ze 
které vybírá tu nejvhodnější akci
\item Intencionální agent - zvažuje své možnosti jak dosáhnout cíle. Je tedy motivován k tomu, aby reagoval co nejlépe. 
\item Sociální (modelový) agent - má k dispozici modely chování 
ostatních agentů a může je tak využívat ke svému rozhodování
\item Hybridní agent - může vykazovat vlastnosti výše zmíněných typů 
agentů podle nastalé situace
\end{itemize}

Typy agentů dle ZUM:
\begin{itemize}
\item izolovaní - ekvivalent single-agentního systému
\item kooperativní - jednající společně za účelem dosažení týmového cíle
\item self-interested - každý maximalizuje své vlastní dobro, zvažuje ostatní agenty
\item kombinace výše uvedených - spolupracující s jedněmi a soutěžící s jinými agenty
\end{itemize}

\section{Obecné vlastnosti kladené na agenty}
\begin{enumerate}
\item autonomní - pracuje bez zásahu z vnějšku, do jisté míry řídí své akce,
\item sociální - interaguje s ostatními agenty,
\item reaktivní - vnímá okolí a reaguje na jeho změny,
\item proaktivní - sám se iniciativně pouští do řešení problémů
\item mobilní - působit aspoň virtuálně na různých místech. Umístění v konkrétním místě prostředí je považováno za důležitou vlastnost. 
\item korektní - neposkytovat vědomě nepravdivé informace
\item benevolentní - snažit se udělat to, o co je žádán,
\item racionální -  jednat tak, aby dosáhl svých cílů.
\end{enumerate}

\section{Důležité další vlastnosti agentů}
\begin{enumerate}
\item vysílat a přijímat informace od ostatních agentů,
\item zpracovávat informace a činit z nich závěry, tj. brát je v úvahu při dalším rozhodování
\item provádět akce. Ty se mohou navíc časem měnit.
\item uvažovat o schopnostech ostatních agentů,
\item generovat cíle a plány pro sebe i ostatní agenty,
\item kooperovat s ostatními, tj. vyjednávat o rozdělení úkolů, činit  závazky a uzavírat smlouvy
\item modifikovat a aktualizovat modely chování.
\end{enumerate}

\section{Požadavky na agenty}
Fyzická struktura spolu s vlastnostmi a požadavky kladenými na jednotlivé části (agenti + komunikační síť).
\begin{itemize}
\item pokrytí, tj. je třeba zajistit, aby každé podúloze byla přidělena skupina agentů, která ji dovede řešit,

\item spojení, tj. aby agenti si byli schopni mezi sebou předávat informace, tj. byli schopni kooperovat a vyměňovat si dílčí výsledky,

\item výpočetní potenciál, tj. dostatečné možnosti agentů daný problém zvládnout
\end{itemize}

\section{Organizace multiagentního systému}
\begin{itemize}
\item Kooperace - schopnosti jednotlivých agentů vzájemně spolupracovat (sdělování dílčích výsledků, plánů, získávání informací) 
\item
Kompetice
(soutěživost mezi agenty) 
-
ostatní agenti jsou 
bráni za konkurenty. Snaha maximalizovat svůj vlastní zisk.
\item
Koordinace
(součinnost kooperujících agentů) 
-
zvýšit 
pravděpodobnost a rychlosti dosažení stanovených cílů. 
Koordinace realizuje kooperaci.
\item
Vyjednávání
–
cílem je odstranit rozpory mezi agenty, 
vyvarovat se konfliktům, maximálně využít dostupné zdroje 
apod. Vyjednávací protokol. Kooperujících i kompetiční 
agenti.
\item 
Komunikace
-
způsob předávání informací mezi agenty. Je 
prostředkem zejména pro kooperaci agentů.
\end{itemize}


\section{Nepřímá komunikace - metoda tabule}
Tabule je datová struktura, ve které se uchovávají informace 
poskytované jednotlivými agenty a zároveň zdroj informací 
pro ostatní agenty. Tabule se skládá ze dvou částí
\begin{enumerate}
\item datová část
tabule 
-
veškeré informace související s řešeným 
problémem (vstupní data, dílčí mezivýsledky, alternativní 
výsledky, konečná řešení). Obsah tabule datové části tabule 
se mění podle příspěvků jednotlivých agentů.

\item řídící část
tabule 
-
indikace položek pro zpracování, případně 
za aktivaci příslušných agentů, kteří jsou pověřeni 
zpracováním dané informace. Dále spravuje seznam priorit, 
ve kterém mají být 
podúlohy
zpracovány, ruší dosažené 
mezivýsledky, sleduje splnění některých ukazatelů (splnění 
nějaké hypotézy) apod.
\end{enumerate}

Obecné zásady při rozdělování úlohy na podúlohy (s pomocí tabule):
\begin{itemize}

\item každý agent provede změny na tabuli, tj. aktualizuje nebo 
poskytne nové informace o dosavadním řešení,

\item
každý agent oznámí jaké 
podúlohy
na tabuli je schopen řešit,

\item
řídící část tabule vybere na základě právě aktualizovaných 
informací v tabuli agenty, kteří jsou schopni řešit uvedené 
podúlohy
a těm tyto 
podúlohy
přiřadit.
\end{itemize}

Nevýhodou využití tabule je značná obecnost. 


\end{document}

















